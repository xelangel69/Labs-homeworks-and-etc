\documentclass[12pt]{article}
\usepackage[utf8]{inputenc}
\usepackage[T2A]{fontenc}
\usepackage[russian]{babel}
\usepackage{geometry}
\usepackage{tikz}
\usepackage{amsmath}
\usepackage{amssymb}
\usepackage{calc}

\geometry{top=2cm, bottom=2cm, left=2cm, right=2cm}

\definecolor{byrneRed}{RGB}{210, 50, 40}
\definecolor{byrneBlue}{RGB}{40, 70, 140}
\definecolor{byrneYellow}{RGB}{240, 170, 40}
\definecolor{byrneBlack}{RGB}{20, 20, 20}

\newcommand{\bLine}[4]{%
    \tikz[baseline=-2pt]{
        \draw[#1, line width=3pt, line cap=round] (0,0) -- (#2,0);
        \node[text=black, font=\scriptsize] at (0, 0.25) {#3};
        \node[text=black, font=\scriptsize] at (#2, 0.25) {#4};
    }
}

\newcommand{\bPerp}{\perp}

\newcommand{\bAngle}[4]{%
    \tikz[baseline=0pt]{
        \fill[#1] (0,0) -- (0:0.5) arc (0:#3:0.5) -- cycle;
        \node[text=black, font=\tiny] at (-0.1, 0) {#4};
        \node[text=black, font=\tiny] at (0.1, 0.6) {#2};
    }%
}

\newcommand{\bSector}[3]{%
    \tikz[baseline=-2pt]{
        \fill[#1] (0,0) -- (#2:0.5) arc (#2:#3:0.5) -- cycle;
    }%
}

\newcommand{\bSemi}{%
    \tikz[baseline=-2pt]{
        \draw[black, line width=1.5pt] (0.5,0) arc (0:180:0.5) -- cycle;
        \draw[black, line width=1pt] (0,0) -- (0, 0.5);
    }%
}

\newcommand{\DropCapE}{%
    \begin{tikzpicture}[baseline=1cm]
        \fill[black] (0,0) rectangle (2.2, 2.2);
        \draw[white, very thick] (0.2, 0.2) rectangle (2.0, 2.0);
        \node[white, font=\fontsize{60}{60}\selectfont, anchor=center] at (1.1, 1.1) {\textbf{E}};
        \draw[white, thin] (0.2,0.2) to[out=45, in=225] (2,2);
        \draw[white, thin] (0.2,2) to[out=-45, in=135] (2,0.2);
    \end{tikzpicture}
}

\begin{document}

\noindent
{\large \textbf{КНИГА I ПРЕДЛ. XIII. ТЕОРЕМА}} \hfill {\large \textbf{37}}

\vspace{1cm}

\begin{minipage}[t]{0.65\textwidth}
    
    \noindent
    \makebox[0pt][l]{\DropCapE}
    \hspace{2.4cm}
    \begin{minipage}[t]{\linewidth-2.5cm}
        \textit{сли прямая линия} \bLine{byrneYellow}{1.0}{E}{D} \textit{восставленная на другой прямой линии} \bLine{byrneRed}{1.2}{B}{C} \textit{образует с ней углы, то это будут либо два прямых угла, либо их сумма будет равна двум прямым углам.}
    \end{minipage}

    \vspace{1cm}

    \noindent
    Если \bLine{byrneYellow}{1.0}{E}{D} $\mathbf{\perp}$ к \bLine{byrneRed}{1.2}{B}{C} тогда,
    
    \vspace{0.3cm}
    
    \hspace{1cm}
    \bSector{byrneYellow}{90}{180} \textbf{и} \bSector{byrneBlue}{0}{90} \textbf{=} \bSemi \hspace{0.2cm} (опр. 10),
    
    \vspace{0.8cm}
    
    \noindent
    но если \bLine{byrneYellow}{1.0}{E}{D} будет не $\mathbf{\perp}$ к \bLine{byrneRed}{1.2}{B}{C},
    
    \noindent
    проведем \bLine{black}{1.0}{A}{D} $\mathbf{\perp}$ \bLine{byrneRed}{1.2}{B}{C} (пр. I.11);
    
    \vspace{0.3cm}
    \hspace{1cm}
    \bSector{byrneYellow}{90}{180} \textbf{+} \bSector{byrneBlue}{0}{90} \textbf{=} \bSemi \hspace{0.2cm} (постр.),
    
    \vspace{0.5cm}
    \hspace{1cm}
    \bSector{byrneYellow}{90}{180} \textbf{=} \bSector{byrneRed}{90}{180} \textbf{=} \bSector{byrneRed}{135}{180} \textbf{+} \bSector{byrneBlue}{0}{60}
    
    \vspace{0.5cm}
    \noindent
    $\therefore$ \hspace{0.5cm}
    \bSector{byrneYellow}{90}{180} \textbf{+} \bSector{byrneBlue}{0}{90} \textbf{=} 
    \bSector{byrneYellow}{90}{180} \textbf{+} \bSector{byrneRed}{0}{90} \textbf{+} \bSector{byrneBlue}{0}{60} \hspace{0.2cm} (акс. II)
    
    \vspace{0.5cm}
    \hspace{1cm}
    \textbf{=} \bSector{byrneYellow}{0}{180} \textbf{+} \bSector{byrneBlue}{0}{60} \textbf{=} \bSemi.

    \vspace{1cm} \hfill \textbf{ч. т. д.}

\end{minipage}
\hfill
\begin{minipage}[t]{0.3\textwidth}
    \vspace{0pt}
    \begin{tikzpicture}[scale=1.2]
        \coordinate (D) at (0,0);
        \coordinate (B) at (-2,0);
        \coordinate (C) at (2,0);
        \coordinate (A) at (0, 3);
        \coordinate (E) at (1.5, 2.5);

        \fill[byrneYellow] (D) -- (-0.6,0) arc (180:60:0.6) -- cycle;
        
        \fill[byrneBlue] (D) -- (0.6,0) arc (0:60:0.6) -- cycle;
        
        \draw[byrneRed, line width=3pt] (B) -- (C);
        \draw[black, line width=3pt] (D) -- (A);
        \draw[byrneYellow, line width=3pt] (D) -- (E);
        
        \node[below] at (D) {\scriptsize D};
        \node[below] at (B) {\scriptsize B};
        \node[below] at (C) {\scriptsize C};
        \node[above] at (A) {\scriptsize A};
        \node[right] at (E) {\scriptsize E};
        
    \end{tikzpicture}
\end{minipage}

\end{document}